\section{EKLER}

\subsection{Proje Kaynak Kodları ve Çalışan Sistem}

Bu projede geliştirilen sistemin kaynak kodlarına ve çalışan web uygulamasına aşağıdaki linklerden erişilebilir:

\subsubsection{GitHub Repository}
Projenin tüm kaynak kodları, dokümantasyon ve geliştirme tarihçesi:

\textbf{Repository URL:} \url{https://github.com/zubeyirtosun/coraza-log-som}

Bu repository'de yer alan ana bileşenler:
\begin{itemize}
    \item Python kaynak kodları (data\_processing.py, advanced\_clustering.py, visualizations.py)
    \item Streamlit web uygulaması (main.py)
    \item Gereksinim dosyaları (requirements.txt, packages.txt)
    \item Örnek log verileri (logFiles/)
    \item Dokümantasyon ve README dosyaları
\end{itemize}

\subsubsection{Çalışan Web Uygulaması}
Sistemin canlı demo versiyonuna Streamlit Cloud üzerinden erişim:

\textbf{Uygulama URL:} \url{https://coraza-log-som.streamlit.app/}

Web uygulaması özellikleri:
\begin{itemize}
    \item Gerçek zamanlı log analizi
    \item İnteraktif SOM görselleştirmesi
    \item Meta-kümeleme algoritmaları karşılaştırması
    \item PDF rapor üretimi
    \item Boyut indirgeme analizi (PCA, t-SNE, UMAP)
\end{itemize}

\newpage

\subsection{Sistem Kullanım Kılavuzu}

\subsubsection{Yerel Kurulum}
\begin{enumerate}
    \item Repository'yi klonlayın: 
    
    \texttt{git clone https://github.com/zubeyirtosun/coraza-log-som}
    \item Gerekli paketleri yükleyin: \texttt{pip install -r requirements.txt}
    \item Uygulamayı başlatın: \texttt{streamlit run main.py}
\end{enumerate}

\subsubsection{Web Uygulaması Kullanımı}
\begin{enumerate}
    \item Tarayıcınızda \url{https://coraza-log-som.streamlit.app/} adresini açın
    \item Sol menüden analiz türünü seçin
    \item JSON formatında log dosyanızı yükleyin veya örnek veri kullanın
    \item SOM parametrelerini ayarlayın
    \item Analiz sonuçlarını ve görselleştirmeleri inceleyin
    \item İsteğe bağlı olarak PDF raporu indirin
\end{enumerate} 